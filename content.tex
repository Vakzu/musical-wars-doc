\section*{Этап 1}

\addcontentsline{toc}{section}{Этап 1}

\subsection*{Описание предметной области}

Существует вселенная, где герои живут уже тысячи лет. Новые герои поставляются корпорацией и имеют разный запас здоровья.\\

Есть пользователи, те инопланетные существа, которые покупают героев и зарабатывают на боях своих героев.\\

Герои деруться нотами из песен, с каждым героем корпорация поставляет в комплекте несколько песен. Песня может измельчаться на ноты, своеобразные "патроны", которые наносят урон.

\addcontentsline{toc}{subsection}{Описание предметной области}

\subsection*{Описание бизнес процессов}

\addcontentsline{toc}{subsection}{Описание бизнес процессов}

Игроки покупают героев. У игроков есть валюта, чтобы платить за героев, у героев есть цена. Также у героев есть минимальный порог опыта, который должен иметь игрок, чтобы купить героя.\\

После покупки героев, игроки выставляют по одному герою на поле битвы. Участие может принимать до пяти игроков. Перед сражением игрок может выбрать эффект, с которым будет ходить его герой всю битву. Герои деруться между собой, оружием являются ноты, компоненты песен, которые идут в инвентаре с покупными героями.\\

После того, как разыгралось сражение и есть выигравший герой. У всех игроков отнимается опыт и перераспределяется в пользу игроков, проранжированных по местам. Также первому месту дается вознаграждение в игровой валюте. \\

Герои могут быть заблокированы, если ведуться какие-то работы на сервере.

\subsection*{Сущности}

\addcontentsline{toc}{subsection}{Сущности}

\begin{enumerate}
    \item Песня(id, имя, id\_героя)
    \item Нота(id, имя, урон)
    \item Герой(id, имя, цена, порог\_по\_опыту, здоровье, доступно\_для\_покупки)
    \item Игрок(id, имя, баланс, опыт)
    \item Личность(id, id\_героя, id\_игрока)
    \item Участник\_драки(id\_личности, id\_драки, id\_эффекта\_в\_драке)
    \item Эффект(id, название, цена)
    \item Драка(id, время\_начала, время\_конца, id\_локации)
    \item Локация(id, название)
\end{enumerate}

\section*{Этап 2}

\addcontentsline{toc}{section}{Этап 2}

\subsection*{Нарисовать ER-диаграмму предметной области}

\addcontentsline{toc}{subsection}{Нарисовать ER-диаграмму предметной области}

% Text

\subsection*{На основе ER-модели построить даталогическую модель}

\addcontentsline{toc}{subsection}{На основе ER-модели построить даталогическую модель}

% Text